\section{Chapter 8}
\begin{enumerate}

% 8.25 %
\item[8.25] An undirected graph is \textbf{bipartite} if its nodes may be divided
into two sets so that all edges go from a node in one set to a node in the other set.
Show that a graph is bipartite if and only if it doesn't contain a cycle that has
an odd number of nodes.

Let $BIPARTITE = \{ \langle G \rangle | \ G$ is a bipartite graph \}. Show that
$BIPARTITE \in$ NL.
\\
\textbf{Solution:}
Let $G=(V,E)$ be bipartite.

So, let $V=A \cup B$ such that $A \cap B = \emptyset$ and that all edges $e \in E$
are such that $e$ is of the form $\{a,b\}$ where $a \in A$ and $b \in B$.

(This is the definition of a bipartite graph.)

Suppose $G$ has (at least) one odd cycle $C$.

Let the length of $C$ be $n$.

Let $C=(v_1,v_2,…,v_n,v_1)$.

WLOG, let $v_1 \in A$. It follows that $v_2 \in B$ and hence $v_3 \in A$, and so on.

Hence we see that $\forall k \in \{1,2,…,n\}$, we have:

:$v_k \in \begin{cases}
A : & k \text{ odd} \\
B : & k \text{ even}
\end{cases}$

But as $n$ is odd, $v_n \in A$.

But $v_1 \in A$, and $(v_n,v_1) \in C_n$.

So $(v_n, v_1) \in E$ which contradicts the assumption that $G$ is bipartite.

Hence if $G$ is bipartite, it has no odd cycles.

\textbf{Note}: This only proves one direction of the iff, to complete the proof we will need
to show that having no odd cycles implies that a graph is bipartite and show that
$BIPARTITE \in$ NL.

\end{enumerate}
