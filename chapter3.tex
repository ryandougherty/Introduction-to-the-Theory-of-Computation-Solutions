\section{Chapter 3}
\begin{enumerate}

%		3.3		%
\item[3.3]Modify the proof of Theorem 3.16 to obtain Corollary 3.19, showing that a language is decidable iff some nondeterministic Turing machine decides it. (You may assume the following theorem about trees. If every node in a tree has finitely many children and every branch of the tree has finitely many nodes, the tree itself has finitely many nodes.)
\\
\textbf{Solution:} \alreadyanswered

%		3.5		%
\item[3.5]Examine the formal definition of a Turing machine to answer the following questions, and explain your reasoning.
\\
\textbf{Solution:} \alreadyanswered

%		3.10		%
\item[3.10]Say that a \textbf{\emph{write-once Turing machine}} is a single-tape TM that can alter each tape square at most once (including the input portion of the tape). Show that this variant Turing machine model is equivalent to the ordinary Turing machine model. (Hint: As a first step, consider the case whereby the Turing machine may alter each tape square at most twice. Use lots of tape.)
\\
\textbf{Solution:} \alreadyanswered

%		3.15		%
\item[3.15]Show that the collection of decidable languages is closed under the operation of
\begin{enumerate}
\item[a.]union.
\\
\textbf{Solution:} \alreadyanswered
\end{enumerate}

%		3.16		%
\item[3.16]Show that the collection of Turing-recognizable languages is closed under the operation of
\begin{enumerate}
\item[a.]union.
\\
\textbf{Solution:} \alreadyanswered
\end{enumerate}

%		3.22		%
\item[3.22]Let $A$ be the language containing only the single string $s$, where $s = 0$ if life never will be found on Mars, and $s = 1$ if life will be found on Mars someday. Is $A$ decidable? Why or why not? For the purposes of this problem, assume that the question of whether life will be found on Mars has an unambiguous YES or NO answer.
\\
\textbf{Solution:} \alreadyanswered

\end{enumerate}